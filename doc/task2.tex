\documentclass[11pt, a4paper]{article}

\usepackage[margin=0.75in]{geometry}
\usepackage{enumerate}

\begin{document}

\title{Task 2: User Programs\\Design Document}
\author{}
\date{}
\maketitle

\begin{center}
Group:\\
Robert Barr, rjb19@ic.ac.uk\\
Sebastian Males, sm919@ic.ac.uk\\
Euan Scott-Watson, es1519@ic.ac.uk\\
Alex Usher, awu19@ic.ac.uk
\end{center}

\section{Argument Passing}

\subsection{Data Structures}
A1: (1 mark)\\
Copy here the declaration of each new or changed 'struct' or 'struct' member,
global or static variable, 'typedef', or enumeration. Identify the purpose
of each in roughly 25 words.
\begin{center}\textbf{moved to thread.h}\end{center}
\begin{verbatim}
#define THREAD_MAGIC 0xcd6abf4b
\end{verbatim}
A random value for \verb|struct thread|'s magic member, used to detect overflow.
Moved to thread.h so it can be accessed in process.c to detect overflow in argument passing.

\subsection{Algorithms}
A2: (2 marks)\\
How does your argument parsing code avoid overflowing the user's stack page?
What are the efficiency considerations of your approach?\bigskip\\
After each write to the stack, a call to \verb|check_thread_magic| is made. This checks whether an overflow
has occurred by seeing if the current thread's 'magic' field has changed from \verb|THREAD_MAGIC|. If the 'magic'
value has changed, the allocated pages are freed and the thread exits. All other memory allocations in the
argument parsing code use \verb|palloc_get_page| to get their own page from memory and so do not use
the user stack page. This implementation is not the most time or space efficient as multiple pages in
memory are used up instead of just one and \verb|palloc_get_page| acquires a lock which means the thread may
have to wait for the lock to be released first, however it avoids any chance of overflowing the stack page.

\subsection{Rationale}
A3: (2 marks)\\
Pintos does not implement \verb|strtok()| because it is not thread safe. Explain the
problem with \verb|strtok()| and how \verb|strtok_r| avoids this issue.\bigskip\\
\verb|strtok()| uses a statically allocated buffer which could cause a race condition
if multiple threads attempt to call \verb|strtok()| at the same time as one thread could
overwrite the buffer whilst the other was attempting to write to it. \verb|strtok_r()|
avoids this issue by using a dynamically allocated buffer so that every call to \verb|strtok_r()|
uses a different location in memory (using the saveptr) and so there is no chance of one thread overwriting the buffer
of another thread calling \verb|strtok_r()|.\bigskip\\
A4: (3 marks)\\
In Pintos, the kernel separates commands into an executable name and arguments.
In Unix-like systems, the shell does this separation. Identify three advantages
of the Unix approach.\bigskip\\
One advantage of the Unix approach is that it provides increased security as
the user cannot access any kernel code before execution and so cannot exploit any bugs in the argument
passing code.
A second advantage of the Unix approach it is the responsibility of the shell to parse the arguments
correctly and so the kernel code can be used in different shell types instead of having to modify the
kernel code for each different shell. This means each user can use a different shell interpreter for the same kernel code.
A third advantage of the Unix approach is that users may only access the files and commands they have
permission to access as the shell runs as the user and so will reject invalid access before
reaching the kernel code. This stops the kernel from incorrectly giving access to a resource the user
doesn't have permissions for and allows different users to have different permissions.

\section{System Calls}
\subsection{Data Structures}
B1: (6 marks)\\
Copy here the declaration of each new or changed 'struct' or 'struct' member, 
global or static variable, 'typedef', or enumeration. Identify the purpose of each
in roughly 25 words.

\begin{center}\textbf{thread.h}\end{center}

\begin{verbatim}
#define FD_START 2 /* First fd for each thread */
\end{verbatim}
\verb|FD_START| is the minimum value a file descriptor can take, since 0 is taken for 
\verb|STDIN_FILENO| and 1 is taken for \verb|STDOUT_FILENO|. We initialise \verb|next_fd| to this
value in \verb|thread.c|.\bigskip\\
Added to 'struct' thread:
\begin{verbatim}
  struct list open_files;  /* list containing files in use by the thread. */
  struct list children;    /* List of child threads. */
  struct child_elem *duty; /* Pointer to the child duties. Can be NULL. */    
  struct file *open_exec;  /* Executable file being run by a thread. Can be NULL */
  int next_fd;             /* Stores the next fd to give to a file being opened. */
\end{verbatim}
\verb|open_files| is a list containing all files currently in use by the thread.\\
\verb|children| stores the list of child threads, whereby the given thread 
is the parent of all elements in this list.\\
\verb|duty| points to the thread's corresponding third-party struct that would be inserted
into a parent thread's \verb|children| list (see below). It is NULL if the thread is not a
child, or is deallocated and set to NULL upon termination of the parent process.\\
\verb|open_exec| stores a user program's executable file, used for denying and allowing
writes; this is freed
when the process exits.\\
\verb|next_fd| stores the next file descriptor for this particular process. It
is initialised to \verb|FD_START| which is the first valid fd value.

\begin{verbatim}
/* The struct used to represent the "childhood" of a process & the "duties"
 a child process has towards its parent. This structure outlives the
 process it is associated with by being malloced. */
struct child_elem { 
  struct semaphore *wait; /* Semaphore used for waking up the parent,
                             can be NULL. */
  int exit_status;        /* The status of the child if it has exited, 
                             INVALID_STATUS otherwise. */
  tid_t thread_tid;       /* The id of the child associated with the child_elem. */
  struct thread *t;       /* Pointer to the associated thread, can be NULL. */
  struct list_elem elem;  /* List_elem to insert into the children list. */
};
\end{verbatim}
This struct represents the third-party structure which we use for \verb|process_wait()|.\\
\verb|wait| and \verb|exit_status| are related. If the child has not terminated and has had
wait called upon it, then \verb|wait| is initialised and the child is responsible for
calling \verb|sema_up()| on it before termination. Upon termination, the child is also
responsible for setting \verb|exit_status()| to its exit status. However, if when
a child exits its \verb|child_elem| is set to NULL, its parent process
has already terminated (or it doesn't exist) hence, the child doesn't have to do
anything - no "duty". \\
\verb|elem| is simply the \verb|list_elem| used to insert this structure into a parent
thread's \verb|children| list.\\
\verb|thread_tid| is the thread id of the child thread. It is important to store
this here since after the termination of the child the member \verb|t| will be NULL
but we might still need the tid for lookups in the children list from
\verb|thread|.\\ 
\verb|t| is the pointer to the child's thread struct. This is not NULL, as long as
the child hasn't finished.

\begin{center}\textbf{syscall.h}\end{center}
\begin{verbatim}
#define MAX_BUFFER_LENGTH 256
#define HANDLED_SYSCALLS 13
#define ERROR -1
#define USER_VADDR_BOTTOM ((void *)0x08048000)
#define SAFE_EXIT NULL

/* Used to insert a file mapping of file descriptor,
 * struct file into a thread's file list. */
struct file_elem {
  struct list_elem elem;
  struct file *file;
  int fd;
};
\end{verbatim}
\verb|MAX_BUFFER_LENGTH| defines the maximum size (in bytes) of a buffer write to \verb|STDOUT_FILENO|.
This follows the specification that any reasonable size (below a few hundred bytes) can be written in
one write.\\
\verb|HANDLED_SYSCALLS| defines the number of syscalls we handle in task, which is important for
initialising the array of syscalls in \verb|syscall.c|.\\
\verb|ERROR| is the error code we return when there is a problem within a syscall function.\\
\verb|USER_VADDR_BOTTOM| represents the lowest address that a user memory location can have. It is
used when validating user memory addresses.\\
We use the type \verb|struct file_elem| to represent a mapping from a \verb|struct file| pointer
to the corresponding file descriptor. This is inserted into a thread's \verb|open_files| list using
\verb|elem|, and is used to obtain the \verb|struct file *| from the file descriptor in \verb|syscall.c|.\\
\verb|SAFE_EXIT| is a handy macro because we have
made our calls to \verb|sys_exit| with NULL as a
parameter to
default to the exit code -1. Hence instead of writing
\verb|sys_exit(NULL)| we write the more explicit \verb|sys_exit(SAFE_EXIT)|.

\begin{center}\textbf{syscall.c}\end{center}
\begin{verbatim}
/* Array of sys call handlers mapped to their specific code.
 * Each function is responsible for retrieving its own parameters
 * from the stack and setting the return result into the eax register.*/
intr_handler_func *handlers[HANDLED_SYSCALLS];

/* Lock to synchronize the file system - called whenever files are used. */
struct lock file_lock;
\end{verbatim}
\verb|handlers| is an array of function pointers, each representing one of the syscalls. They
are indexed following \verb|syscall_nr.h|.\\
\verb|file_lock| is used to synchronise accesses to the file system, avoiding potential data
loss/corruption. This is also defined as \verb|extern| in \verb|syscall.h| to allow its use
in other files.

\subsection{Algorithms}
B2: (2 marks)\\
Describe how your code ensures safe memory access of user provided data from within
the kernel.\bigskip\\
We have two functions: \verb|validate_byte()| and \verb|validate_buffer()|. The first function checks a
pointer is valid by calling \verb|is_user_vaddr()| and calls \verb|get_user| to check for segfaults. If
this comes back as an error, the \verb|safe_exit| function is called. The second function validates
the entire buffer by looping through each value in the buffer. In the loop, it calls the \verb|validate_byte()|
function to check if the data is valid, if not it will exit. \\
There is also \verb|read_mem_user()| which takes in a source for data and a destination for the data to be
copied into. It then loops through the data, validates it using \verb|validate_byte()| and if the data
is valid, writes it into the destination location. \\
Finally, \verb|get_user()| and \verb|put_user()| are two functions given by the spec that are used
to read and write a byte to a virtual address. If these functions fail we know that there was an error.\bigskip\\
B3: (3 marks)\\
Suppose that we choose to verify user provided pointers by validating them before use
(i.e. using the first method described in the spec).\\
What is the least and greatest possible number of inspections of the page table (e.g.
calls to\\ \verb|pagedir_get_page()|) that would need to be made in the following cases?\\
a) A system call that passes the kernel a pointer to 10 bytes of user data.\\
b) A system call that passes the kernel a pointer to a full page (4,096 bytes) of user data.\\
c) A system call that passes the kernel a pointer to 4 full pages (16,384 bytes) of
user data.\\
You must briefly explain the checking tactic you would use and how it applies to each case 
to generate your answers.\bigskip\\
a) For 10 bytes, the least number of calls would be 1. If we were to get a kernel virtual address that has
more than 10 bytes to the end of the page, we know that all the data would be in the page. If the
data is not contiguous then the greatest number of calls we would have to do would be 10 in
which each byte is stored in a different page that we must get. \\
b) If all the data is stored on one page then at minimum it would be called once. The greatest number
of times would be 4096 if the data were to be distributed in byte sized pieces across 4096 different pages
-i.e. called once for each byte. \\
c) As 16,384 bytes of data will be 4 pages worth of data, if all 4 pages worth of data were able to be stored
into 4 pages, then only 4 calls would be required. However, they could be spread like before
with each byte being on a new page. This would therefore result in 16,384 calls to \verb|pagedir_get_page()|.\bigskip\\
B4: (2 marks)\\
When an error is detected during a system call handler, how do you ensure that all 
temporarily allocated resources (locks, buffers etc.) are freed?\bigskip\\
When an error is detected, a function named "\verb|safe_exit()|" is called. This checks that if the current thread is holding
on to the file system lock, then it releases it before killing the thread. Moreover, whenever there is an allocation of memory
it will be freed within the function it was allocated no matter the path taken. This means that if the thread fails, it will
be freed before the thread is killed, and if the thread was successful, then at the end of the function the memory allocated is freed.\bigskip\\
B5: (8 marks)\\
Describe your implementation of the 'wait' system call and how it interacts with
process termination for both the parent and child.\bigskip\\
Each process keeps track of its children with the
\verb|children| list member from thread. This contains
\verb|child_elems| and each child adds itself to the
parent's list at creation in \verb|thread_create|. These
elements are manually allocated and are meant to represent
the duty a child has to its parent. That is, wake it up
on exit if a wait has been called on the child or
set its exit
code so the parent can retrieve it from there. The
\verb|child_elem| of a process is not freed when the
process exits but rather when the parent exits or
after wait has been called on the process (so a double
wait is not possible). Thus, when a process calls
wait on another if the process is indeed a child
of the caller, the parent checks if the exit code
has been set in the \verb|child_elem| from its
\verb|children| list. If it did then return, otherwise
the parent sets itself to the \verb|t| member of the
\verb|child_elem| and blocks so the child knows to wake up its parent upon termination.

\subsection{Synchronisation}
B6: (2 marks)\\
The "exec" system call returns -1 if loading the new executable fails,
so it cannot return before the new executable has completed loading. 
How does your code ensure this? How is the load success/failure
status passed back to the thread that calls "exec"?\bigskip\\
The \verb|sys_exec| call returns the return value of \verb|process_execute|, 
the parent creates a semaphore and a bool called 'success' in \verb|process_execute|
and passes as an aux argument to \verb|thread_create|, which is then passed
to \verb|start_process|, an array of pointers containing the 2 and the file name.
After the \verb|thread_create| call, the parents blocks itself with a \verb|sema_down|
if \verb|thread_create| did not return -1 (in which case, \verb|process_execute|
quits right away by returning -1).
If any of the allocations in \verb|start_process| fail, the child process sets
the bool success to false and does \verb|sema_up| (and if all of the allocations don't
raise any issues then \verb|sema_up| is called subsequently anyway). After the parent
has been woken up, if the boolean success is false then the call returns with a -1.\bigskip\\
B7: (5 marks)\\
Consider parent process P with child process C.\\
How do you ensure proper synchronisation and avoid race conditions where:\bigskip\\
i)   P calls wait(C) before C exits?\\
If P calls wait(C) before C exits and C is indeed a child of P then
P can tell that C hasn't finished because the exit status from the
\verb|child_elem| is set to it's default value. P proceeds to
create a semaphore that is added to \verb|child_elem| 
(so the child can wake up the parent) and
does \verb|sema_down|. Notice that if \verb|sema_up| is called
before from the child before the parent calls \verb|sema_down| (because
of a sudden context switch) this is not an issue since if the value
in the semaphore is 1 \verb|sema_down| doesn't block.\bigskip\\
ii)  P calls wait(C) after C exits?\\
If P calls wait(C) and C is indeed a child of P then the
\verb|child_elem| is still in the \verb|children| list (since this
is freed and removed only by the parent) but the exit status is set
to a different value than its default value. Hence, the parent returns
that without blocking. No synchronisation issues occur since after
the child terminates, only the parent can alter the state of the
\verb|child_elem|.\bigskip\\
iii) P terminates, without waiting, before C exits?\\
If P terminates before C then the child shouldn't do anything on exit.
P frees all the \verb|child_elems| of its children and sets the
pointers from the children process to their duty to NULL. If the child checks
its duty and sees that it is NULL then it doesn't do anything. The important
part to synchronising these 2 is that the checks the child does on the pointer
to \verb|child_elem| are atomic because we turn the interrupts off when this happens
(at the end of \verb|sys_exit| and the end of \verb|process_exit|).\bigskip\\
iv)  P terminates, without waiting, after C exits?\\
If P terminates after C then P simply frees
all of the duties of its children and checks if the child thread is still attached to
the duty. If it isn't (and it this case it isn't because the child thread sets
\verb|t| to NULL when it finished) then it doesn't do anything to it. Notice that
we avoid race conditions here because the interrupts are turned off during this while
loop. Hence, the pointer cannot be set to NULL after it has been checked and before
it is dereferenced.\bigskip\\
Additionally, how do you ensure that all resources are freed regardless of the above
case?\bigskip\\
Fundamentally, the key to freeing all the resources allocated is that the parent allocates
the \verb|child_elems| for all of its children and it frees all of those resources at exit.
The only exception to this is when the parent frees a child earlier after it has waited for it
so that double waits are now allowed but once again, this is safe because the parent is the only
entity managing this memory. It is also worth mentioning again that we ensure synchronisation between
the parent and the child when it comes to the \verb|child_elem| itself by turning the interrupts
off in the child on the sections where it is used so we avoid dereferencing NULL pointers
after checking for it.

\subsection{Rationale}
B8: (2 marks)\\
Why did you choose to implement safe access of user memory from the kernel in this
way?\bigskip\\
We chose to modify the code for \verb|page_fault()| as it would allow the CPU to utilise
its MMU. This meant that we would get faster execution. Therefore, this leads to fewer
bottlenecks when it comes to validating pointers. Moreover, the way in which we implemented
the checks meant that we would not risk leaking any resources as any failed validations
would lead to the freeing of any resources and the file system lock.\bigskip\\
B9: (2 marks)
What advantages and disadvantages can you see to your design for file descriptors?\bigskip\\
Our implementation of file descriptors uses a \verb|struct list| in each thread, storing a
list of structs representing a mapping from a \verb|struct file *| to a file descriptor.
When a file descriptor is referenced, we search through this list to see if there is a
mapping with it. If there is, we can then easily access the corresponding file.\\
We chose this implementation for its benefits concerning time complexity. Although there is
the additional space complexity by storing the list in the thread 'struct', the time
complexity benefits outweigh this factor. The only other plausible solution is to have a
global static list storing the mapping of all file descriptors to their respective files - 
searching through this would evidently take much longer than searching through only those
related to a thread, especially given that this list would then also have to be synchronised.\\
We chose to use an int value, stored within the thread struct, to store the next file
descriptor a newly opened file for the thread would take. We chose to implement this
within the thread struct, as opposed to as a global variable in \verb|syscall.c|.
The main reason behind this is that we can then open a greater amount of
files before the value \verb|next_fd| overflows - a thread would have to open
\verb|INT_MAX| files before exiting, before it would overflow. If we stored this
globally instead, we could only open \verb|INT_MAX| files in total before it would overflow,
and this would not be ideal in a system that may be running for long periods of time.

\end{document}